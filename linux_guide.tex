\documentclass[dvipdfmx,a4paper,11pt]{jsbook}

\usepackage{amsmath,amsfonts}
\usepackage[mathcal]{euscript}
\usepackage{mathtools}
\usepackage{xcolor}
\usepackage{xcoffins,calc}
\usepackage{bm}
\usepackage{amsthm}
\usepackage{amssymb}
\usepackage{pgf}
\usepackage{titlesec}
\usepackage{ifthen}
\usepackage{mathrsfs}
\usepackage[scr]{rsfso}
\usepackage{relsize}
\usepackage{makeidx}
\usepackage{etoolbox}
\usepackage{footnote}
\usepackage[all]{xy}
\usepackage{url}
\usepackage[most]{tcolorbox}

\usepackage[%
dvipdfmx, %欧文ではコメントアウトする
setpagesize=false,
bookmarks=true,
bookmarksdepth=tocdepth,
bookmarksnumbered=true,
colorlinks=true,
linkcolor=blue,
citecolor=blue,
urlcolor=blue,
pdftitle={},
pdfsubject={},
pdfauthor={},
pdfkeywords={}
]{hyperref}


\tcbset {
  base/.style={
    arc=0mm, 
    bottomtitle=0.5mm,
    boxrule=0mm,
    colbacktitle=black!10!white, 
    coltitle=black, 
    fonttitle=\bfseries, 
    left=2.5mm,
    leftrule=1mm,
    right=3.5mm,
    title={#1},
    toptitle=0.75mm, 
  }
}

\tcbset{
    terminalbox/.style={
        colback=black,
        coltext=white,
        boxrule=0pt,
        sharp corners,
        fontupper=\ttfamily,
        boxsep=5pt,
        arc=2mm
    }
}



\definecolor{brandblue}{rgb}{0.34, 0.7, 1}
\newtcolorbox{mainbox}[1]{
  colframe=brandblue,
  breakable=true, 
  base={#1}
}

\newtcolorbox{subbox}[1]{
  colframe=black!30!white,
  breakable=true,
  base={#1}
}

\usepackage{listings,jvlisting} %日本語のコメントアウトをする場合jvlisting(もしくはjlisting)が必要
%ここからソースコードの表示に関する設定
\lstset{
  basicstyle={\ttfamily},
  identifierstyle={\small},
  commentstyle={\smallitshape},
  keywordstyle={\small\bfseries},
  ndkeywordstyle={\small},
  stringstyle={\small\ttfamily},
  frame={tb},
  breaklines=true,
  columns=[l]{fullflexible},
  numbers=left,
  xrightmargin=0zw,
  xleftmargin=3zw,
  numberstyle={\scriptsize},
  stepnumber=1,
  numbersep=1zw,
  lineskip=-0.5ex
}


\usepackage{pxjahyper}
\usepackage{tikz}
\usetikzlibrary{intersections,calc,arrows.meta,arrows}
\usetikzlibrary{hobby}
\usetikzlibrary{decorations.markings}
\usepackage{wrapfig}
\usepackage[truedimen,top=25truemm,bottom=30truemm,hmargin=25truemm]{geometry}
\usepackage{calc}
\usepackage{fancyhdr}
\pagestyle{fancy}
\fancyhead[R]{\nouppercase{\leftmark}} 
\fancyhead[L]{\nouppercase{\rightmark}}

\setcounter{chapter}{0}
\renewcommand{\thechapter}{\Roman{chapter}}

\begin{document}

\setlength{\footskip}{20truemm}

\makeatletter
\newcount\@chars\newcount\@lines
\@chars=40                      % 1行の文字数
\@lines=40                      % 1ページの行数

\newdimen\@kanjiskip
\@kanjiskip=\dimexpr(\textwidth-1zw*\@chars)/\numexpr\@chars-1
\newdimen\@@kanjiskip
\@@kanjiskip=\dimexpr\@kanjiskip/10

\baselineskip=\dimexpr\textheight/\@lines
\kanjiskip=\@kanjiskip plus \@@kanjiskip minus \@@kanjiskip
\parindent=\dimexpr 1zw+2truept
\parindent=\dimexpr\parindent+\@kanjiskip
\makeatother


\title{Linuxサーバーの導入}
\author{山内優弥}
\date{\today}
\maketitle

\setcounter{tocdepth}{2}

\tableofcontents
\clearpage

\chapter{まえがき}

\section{この文書について}
Linuxのディストリビューションの一つであるUbuntuの導入からサーバーとしての利用の手順を
述べて文書である.また
\begin{mainbox}{ああああ}
  あああああ
\end{mainbox}
や
\begin{subbox}{いいいい}
  いいいいい
\end{subbox}
で書かれる文章は必ず読む必要はなく適当な補足であることにする.
\section{実行環境} 
\begin{itemize}
  \item 空っぽのパソコン
  \item WSL2:
  \begin{enumerate}
    \item[] WSLバージョン:2.3.26.0
    \item[] カーネルバージョン:5.15.167.4-1
    \item[] WSLgバージョン:1.0.65
    \item[] MSRDCバージョン:1.2.5620
    \item[] Direct3Dバージョン:1.611.1-81528511
    \item[] DXCoreバージョン:10.0.26100.1-240331-1435.ge-release
    \item[] Windowsバージョン:10.0.26100.2454
  \end{enumerate}
  WSL2内のUbuntuのバージョン:Ubuntu 22.04.5 LTS
\end{itemize}





\chapter{Ubuntuの導入}
\section{Ubuntuのダウンロード}
今回はUbuntu Japanese Teamから日本語環境のUbuntuのダウンロードをこのページ\url{https://www.ubuntulinux.jp/japanese}
からダウンロードします.どのミラーサイトを用いても構いません.
\begin{mainbox}{Ubuntuの日本語環境のバージョンについて}
  今現在(2024/10/21)最新のLTS(Long-Term Support)バージョンはUbuntu 24.04.1 LTSだが,
  2024/06/10の記事(\url{https://www.ubuntulinux.jp/News/ubuntu2404-ja-remix})において
  Ubuntu Japanese TeamがUbuntu 24.04 LTSの日本語Remixをリリースしないことを表明しています.
\end{mainbox}

\begin{subbox}{Ubuntuの歴史}
  Ubuntuは,2004年にリリースされたLinuxディストリビューションで,Debianをベースにしています.
  開発元はCanonicalという会社で,
  創設者のMark Shuttleworthによって立ち上げられました.Ubuntuの
  名称は南アフリカのズールー語で「他者への思いやり」や「人間性」を意味し,
  オープンソースコミュニティやユーザー間での協力を象徴しています.
  \begin{itemize}
    \item 2004年: 最初のバージョン「Ubuntu 4.10 "Warty Warthog"」がリリースされました.
    Debianベースで使いやすさを重視し,デスクトップLinuxの普及を目指しました.
    \item 2006年: LTS(Long Term Support)リリースが導入され,「Ubuntu 6.06 LTS "Dapper Drake"」が最初のLTS版です.
    LTS版は5年間の長期サポートが提供され,企業や組織での採用が進みました.
    \item 2010年: Ubuntuはデフォルトのデスクトップ環境をGNOMEからUnityに変更しました.Unityはユーザーインターフェースを刷新し,
    使いやすさを向上させることを目的としていましたが,一部のユーザーから批判もありました.
    \item 2017年: Unityの開発が中止され,GNOMEデスクトップに戻ることが発表されました.「Ubuntu 17.10 "Artful Aardvark"」
    でUnityからGNOMEへの移行が行われ,従来のデスクトップ環境を採用する形に戻りました.
    \item 2020年代以降: クラウドやIoT(モノのインターネット),サーバー市場でもUbuntuは広く利用されています.また,
    Ubuntuベースの派生ディストリビューション(Kubuntu,Xubuntu,Lubuntuなど)がそれぞれのニーズに応じて発展してきました.
  \end{itemize}
\end{subbox}

\begin{mainbox}{isoファイルについて}
  気が向いたら書く.
\end{mainbox}


\section{rufus}

rufusとはブータブルUSB作成するソフトウェアで次のURLからダウンロードできる.
\url{https://rufus.ie/ja/#google_vignette}\\
USBを挿してrufusを開くと次のようなウィンドウが出現する.
\begin{figure}[htbp]
  \begin{center}
    \includegraphics[width = 50mm]{rufus.png}
    \caption{rufusのウィンドウ}
  \end{center}
\end{figure}
\\
「選択」から先程ダウンロードしたUbuntuのisoファイルを選択して,スタートを選択する.これでUbuntuを起動する準備が整った.

\section{Ubuntuとの邂逅}
isoファイルを入れたUSBを挿し込んでBIOSを起動する.
そして,Bootの順位を変更してUbuntuを一番上に変更する.
これによりUbuntuが起動する.\\
まず初めに現時点では,Ubuntuに
IMEが導入されていないため日本語が入力できない.
これを解消するためにMozcをダウンロードしよう.\\
ターミナルを開いて以下を実行するとダウンロードが完了する.

\begin{tcolorbox}[terminalbox]
  \$ sudo apt update\\
  \$ sudo apt -y install ibus-mozc
\end{tcolorbox}


\begin{subbox}{Mozcについて}

\end{subbox}


\begin{subbox}{-yについて}

\end{subbox}

\section{エディタ}
UNIX系のOSでは,viエディタが標準で搭載されている.
viエディタの基本的な使い方を紹介しよう.
\begin{tcolorbox}[terminalbox]
  \$ vi \textcolor{red}{ファイル名}
\end{tcolorbox}
\noindent とすれば,エディタが起動する.\\
viエディタは二つのモードが存在する.
\begin{center}
  \textbf{コマンドモード}$\rightleftarrows$ \textbf{インサートモード}
\end{center}
はじめに起動時には
\textbf{コマンドモード}になっている.
この状態では,テキストファイル内の文字列の編集はできず,入力された
文字はコマンドとして認識される.\\
とりあえず,覚えておくべきコマンドを紹介しよう.
\begin{enumerate}
  \item カーソル移動
  \begin{itemize}
    \item h\ :\ カーソルを一つ左に
    \item j\ :\ カーソルを一つ下に
    \item k\ :\ カーソルを一つ上に
    \item l\ :\ カーソルを一つ右に
    \item Ctrl+f\ : \ 次のページへ
    \item Ctrl+b\ : \ 前のページへ
  \end{itemize}
  \item 切り取り/コピー/貼り付け
  \begin{itemize}
    \item x\ :\ カーソル位置の文字を切り取る(ほとんどDeleteの意味)
    \item XX\ :\ カーソル位置の手前の文字を切り取る(ほとんどBackspaceの意味)
    \item dd\ :\ カーソルのある行の切り取り
    \item yy\ :\ カーソルのある行のコピー
    \item p\ :\ カーソルのある行の下に貼り付け
    \item P\ :\ カーソルのある行に貼り付け
  \end{itemize}
  \item ファイルの保存や終了など
  \begin{itemize}
    \item :w\ :\ ファイルの保存
    \item :w ファイル名\ :\ 名前をつけてファイルを保存
    \item :q\ :\ エディタの終了
    \item :wq\ :\ :wをして:qするコマンド(i.e. ファイルを保存して終了)
    \item !\ :\ 強制するコマンド.:q! $\leftarrow$ 強制終了, :wq! $\leftarrow$ 強制的に保存して終了
  \end{itemize}
\end{enumerate}
もちろん\textbf{インサートモード}に移行するコマンドが存在する.
インサートモードとはテキストを編集するモードのことである.
\begin{itemize}
  \item i\ :\ カーソルの位置からインサートモードに移行する.
  \item a\ :\ カーソルの後ろからインサートモードに移行する.
  \item o\ :\ カーソルのある行の下に空白行を追加してインサートモードに移行する.
\end{itemize}


\begin{subbox}[viとvim]

\end{subbox}



\chapter{Webサーバー}
\section{Apache}
本書では,WebサーバーとしてApacheを紹介する.\\






\appendix
\chapter{コマンド一覧}
気が向けばコマンドについての一覧を作る.


\end{document}
